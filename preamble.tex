
% Balance the last page using the balance package (see https://ctan.org/pkg/balance)
% Alternative to balance is the pbalance package (see https://ctan.org/pkg/pbalance), which sometimes works better
\usepackage{balance}
%\usepackage{pbalance}

\usepackage{iftex}
% backticks (`) are rendered as such in verbatim environments.
% See following links for details:
%   - https://tex.stackexchange.com/a/341057/9075
%   - https://tex.stackexchange.com/a/47451/9075
%   - https://tex.stackexchange.com/a/166791/9075
\usepackage{upquote}

% Set English as language and allow to write hyphenated"=words
%
% Even though `american`, `english` and `USenglish` are synonyms for babel package (according to https://tex.stackexchange.com/questions/12775/babel-english-american-usenglish), the llncs document class is prepared to avoid the overriding of certain names (such as "act." -> "Abstract" or "Fig." -> "") when using `english`, but not when using the other 2.
% english has to go last to set it as default language
\usepackage[ngerman,main=english]{babel}
%
% Hint by http://tex.stackexchange.com/a/321066/9075 -> enable "= as dashes
\addto\extrasenglish{\languageshorthands{ngerman}\useshorthands{"}}

% Links behave as they should. Enables "\url{...}" for URL typesettings.
% Allow URL breaks also at a hyphen, even though it might be confusing: Is the "-" part of the address or just a hyphen?
% See https://tex.stackexchange.com/a/3034/9075.
\usepackage[hyphens]{url}

% When activated, use text font as url font, not the monospaced one.
% For all options see https://tex.stackexchange.com/a/261435/9075.
\urlstyle{same}

% Improve wrapping of URLs - hint by http://tex.stackexchange.com/a/10419/9075
\makeatletter
\g@addto@macro{\UrlBreaks}{\UrlOrds}
\makeatother

% nicer // - solution by http://tex.stackexchange.com/a/98470/9075
% DO NOT ACTIVATE -> prevents line breaks
%\makeatletter
%\def\Url@twoslashes{\mathchar`\/\@ifnextchar/{\kern-.2em}{}}
%\g@addto@macro\UrlSpecials{\do\/{\Url@twoslashes}}
%\makeatother

% Modern replacement for Times.sty that sets math fonts correctly and scales helvet properly
\usepackage{mathptmx}
\usepackage[scaled=.90]{helvet}
\usepackage{courier}

%% !! Include math libraries before unicode-math
\usepackage{amsmath}
\interdisplaylinepenalty=2500 
\usepackage{amssymb}
\usepackage{mathtools}
\usepackage{lualatex-math}
\usepackage{xfrac}
\usepackage{mdwmath}
\usepackage{cases}


%% !!! If you change the font, be sure that words such as "workflow" can
%% !!! still be copied from the PDF. If this is not the case, you have
%% !!! to use glyphtounicode. See comment at cmap package.
%%
%% Background: "workflow" contains "fl" which is a ligature, which in turn
%%             is rendered as one character in the PDF and needs to be split
%%             whily copying.

\ifluatex
  \usepackage[no-math]{fontspec}
  \defaultfontfeatures{Ligatures = TeX}

  \usepackage{unicode-math}

  % Enable proper ligatures
  % For more information see https://ctan.org/pkg/selnolig
  % language "english" or "ngerman" is passed to selnolig by the document class
  \usepackage{selnolig}
  
  \setmonofont[Scale=0.82]{Cascadia Mono}
%  \setmonofont{Courier New}
%  \setmainfont{Times New Roman}
%  \setmainfont{TeX Gyre Termes}
%    \setmainfont{Times}
    \setmathfont{STIX Two Math}[
    Extension={.otf},
    Path=./stixfonts/fonts/static_otf/,
    Scale=1]
    \setmainfont{Stix Two Text}[
    Extension={.otf},
    Path=./stixfonts/fonts/static_otf/,
    UprightFont={*-Regular},
    BoldFont={*-Bold},
    ItalicFont={*-Italic},
    BoldItalicFont={*-BoldItalic}]
\else
\defaultfontfeatures{Ligatures = TeX, Mapping = tex-text}
  % use nicer font for code
  \usepackage[zerostyle=b,scaled=.75]{newtxtt}

  % Has to be loaded AFTER any font packages. See https://tex.stackexchange.com/a/2869/9075.
  \usepackage[T1]{fontenc}
\fi

\usepackage{ninecolors}

% Character protrusion and font expansion. See http://www.ctan.org/tex-archive/macros/latex/contrib/microtype/


\usepackage[section]{placeins}

% \texttt{test -- test} keeps the "--" as "--" (and does not convert it to an en dash)
\ifLuaTeX
\usepackage[
babel=true, % Enable language-specific kerning. Take language-settings from the languge of the current document (see Section 6 of microtype.pdf)
expansion=alltext,
protrusion=alltext-nott, % Ensure that at listings, there is no change at the margin of the listing
% In the standard configuration, this template is always in the final mode, so this option only makes a difference if "pros" use the draft mode
final % Always enable microtype, even if in draft mode. This helps finding bad boxes quickly.
]{microtype}
\DisableLigatures{encoding = T1, family = tt* }
\else\ifPDFTeX
\usepackage[
babel=true, % Enable language-specific kerning. Take language-settings from the languge of the current document (see Section 6 of microtype.pdf)
expansion=alltext,
protrusion=alltext-nott, % Ensure that at listings, there is no change at the margin of the listing
% In the standard configuration, this template is always in the final mode, so this option only makes a difference if "pros" use the draft mode
final % Always enable microtype, even if in draft mode. This helps finding bad boxes quickly.
]{microtype}
\DisableLigatures{encoding = T1, family = tt* }
\fi
\fi


%\DeclareMicrotypeSet*[tracking]{my}{ font = */*/*/sc/* }%
%\SetTracking{ encoding = *, shape = sc }{ 45 }
% Source: http://homepage.ruhr-uni-bochum.de/Georg.Verweyen/pakete.html
% Deactiviated, because does not look good

\usepackage{graphicx}

% Diagonal lines in a table - http://tex.stackexchange.com/questions/17745/diagonal-lines-in-table-cell
% Slashbox is not available in texlive (due to licensing) and also gives bad results. Thus, we use diagbox
\usepackage{diagbox}

\ifluatex
  \usepackage{spelling}
  \spellingoutput{off}
\fi

\usepackage[dvipsnames, table]{xcolor}
% Code Listings
\usepackage{listings}

\definecolor{eclipseStrings}{RGB}{42,0,255}
\definecolor{eclipseKeywords}{RGB}{127,0,85}
\definecolor{spiceComment}{RGB}{0,128,0}
\definecolor{spiceDotCommand}{RGB}{0,0,255}
\definecolor{spiceContinuationText}{RGB}{255,0,0}
\colorlet{numb}{magenta!60!black}
\colorlet{back-color}{gray9!30!white}

\definecolor{main-color}{rgb}{0.6627, 0.7176, 0.7764}


% SPICE definition
% CUSTOM
\lstdefinelanguage{spice}{
	escapeinside={;(*}{*)},
	basicstyle=\normalfont\scriptsize\ttfamily,
	commentstyle=\color{spiceComment}, % style of comment
	numbers=left,
	stringstyle=\textrm\color{eclipseKeywords}, % style
	string=[s]{"}{"},
	numberstyle=\tiny\ttfamily\color{black},
%	showspaces=true,
%	showtabs=true,
	stepnumber=1,
	breaklines=true,
	breakatwhitespace=true,
	columns=fixed,
	numbersep=8pt,
%	showstringspaces=true,
	breaklines=true,
%	frame=lines,
	backgroundcolor=\color{back-color},
	sensitive=false,
	comment=[l]{\;},
	morecomment=[l]{*},
	moredelim=[l][\color{spiceContinuationText}]{+},
	moredelim=[l][\color{spiceDotCommand}]{.},
}


% JSON definition
% Source: https://tex.stackexchange.com/a/433961/9075

\lstdefinelanguage{json}{
  basicstyle=\normalfont\footnotesize\ttfamily,
  commentstyle=\color{eclipseStrings}, % style of comment
  stringstyle=\color{eclipseKeywords}, % style of strings
  numbers=left,
  numberstyle=\scriptsize,
  stepnumber=1,
  numbersep=8pt,
  showstringspaces=false,
  breaklines=true,
  frame=lines,
%  backgroundcolor=\color{gray}, %only if you like
  string=[s]{"}{"},
  comment=[l]{:\ "},
  morecomment=[l]{:"},
  literate=
    *{0}{{{\color{numb}0}}}{1}
    {1}{{{\color{numb}1}}}{1}
    {2}{{{\color{numb}2}}}{1}
    {3}{{{\color{numb}3}}}{1}
    {4}{{{\color{numb}4}}}{1}
    {5}{{{\color{numb}5}}}{1}
    {6}{{{\color{numb}6}}}{1}
    {7}{{{\color{numb}7}}}{1}
    {8}{{{\color{numb}8}}}{1}
    {9}{{{\color{numb}9}}}{1}
}



\lstset{
  showstringspaces=false,
  extendedchars=true,
  basicstyle=\footnotesize\ttfamily,
  commentstyle=\slshape
  numberstyle=\tiny\ttfamily\color{black},
  stepnumber=1,
  breaklines=true,
  breakatwhitespace=true,
  columns=fixed,
  numbersep=8pt,
  breaklines=true,
  frame=lines,
  tabsize=2,                  % Groesse von Tabs
  numbers=left,
  basewidth=.5em,
  xleftmargin=.5cm,
  % aboveskip=0mm,
  % belowskip=0mm,
  captionpos=b
}


\lstloadlanguages{% Check dokumentation for further languages...
  %[Visual]Basic
  %Pascal
  %C
  %C++
  %XML
  %HTML
  Python,
  Octave
}



\definecolor{checkboxBorder}{RGB}{145,145,145}%"#919191"
\definecolor{editorBackground}{RGB}{255,255,255}%"#FFFFFF"
\definecolor{editorForeground}{RGB}{0,0,0}%"#000000"
\definecolor{editorInactiveselectionbackground}{RGB}{229,235,241}%"#E5EBF1"
\definecolor{editorindentguideBackground1}{RGB}{211,211,211}%"#D3D3D3"
\definecolor{editorindentguideActivebackground1}{RGB}{147,147,147}%"#939393"
\definecolor{editorSelectionhighlightbackground}{RGB}{173,214,255}%"#ADD6FF80"
\definecolor{editorsuggestwidgetBackground}{RGB}{243,243,243}%"#F3F3F3"
\definecolor{activitybarbadgeBackground}{RGB}{0,122,204}%"#007ACC"
\definecolor{sidebartitleForeground}{RGB}{111,111,111}%"#6F6F6F"
\definecolor{listHoverbackground}{RGB}{232,232,232}%"#E8E8E8"
\definecolor{menuBorder}{RGB}{212,212,212}%"#D4D4D4"
\definecolor{inputPlaceholderforeground}{RGB}{118,118,118}%"#767676"
\definecolor{searcheditorTextinputborder}{RGB}{206,206,206}%"#CECECE"
\definecolor{settingsTextinputborder}{RGB}{206,206,206}%"#CECECE"
\definecolor{settingsNumberinputborder}{RGB}{206,206,206}%"#CECECE"
\definecolor{statusbaritemRemoteforeground}{RGB}{255,255,255}%"#FFF"
\definecolor{statusbaritemRemotebackground}{RGB}{22,130,93}%"#16825D"
\definecolor{portsIconrunningprocessforeground}{RGB}{54,148,50}%"#369432"
\definecolor{sidebarsectionheaderBackground}{RGB}{0,0,0}%"#0000"
\definecolor{sidebarsectionheaderBorder}{RGB}{97,97,97}%"#61616130"
\definecolor{tabSelectedforeground}{RGB}{51,51,51}%"#333333b3"
\definecolor{tabSelectedbackground}{RGB}{255,255,255}%"#ffffffa5"
\definecolor{tabLastpinnedborder}{RGB}{97,97,97}%"#61616130"
\definecolor{notebookCellbordercolor}{RGB}{232,232,232}%"#E8E8E8"
\definecolor{notebookSelectedcellbackground}{RGB}{200,221,241}%"#c8ddf150"
\definecolor{statusbaritemErrorbackground}{RGB}{199,46,15}%"#c72e0f"
\definecolor{listActiveselectioniconforeground}{RGB}{255,255,255}%"#FFF"
\definecolor{listFocusandselectionoutline}{RGB}{144,194,249}%"#90C2F9"
\definecolor{terminalInactiveselectionbackground}{RGB}{229,235,241}%"#E5EBF1"
\definecolor{widgetBorder}{RGB}{212,212,212}%"#d4d4d4"
\definecolor{actionbarToggledbackground}{RGB}{221,221,221}%"#dddddd"
\definecolor{diffeditorUnchangedregionbackground}{RGB}{248,248,248}%"#f8f8f8"
\definecolor{newOperator}{RGB}{0,0,255}%"#0000ff"
\definecolor{stringLiteral}{RGB}{163,21,21}%"#a31515"
\definecolor{customLiteral}{RGB}{0,0,0}%"#000000"
\definecolor{numberLiteral}{RGB}{9,134,88}%"#098658"
\definecolor{variableLegacyBuiltinPython}{RGB}{0,0,0}%"#000000ff"
\definecolor{metaDiffHeader}{RGB}{0,0,128}%"#000080"
\definecolor{comment}{RGB}{0,128,0}%"#008000"
\definecolor{constantLanguage}{RGB}{0,0,255}%"#0000ff"
\definecolor{keywordOperatorMinusExponent}{RGB}{9,134,88}%"#098658"
\definecolor{constantRegexp}{RGB}{129,31,63}%"#811f3f"
\definecolor{cssTagsInSelectorsXmlTags}{RGB}{128,0,0}%"#800000"
\definecolor{entityNameSelector}{RGB}{128,0,0}%"#800000"
\definecolor{entityOtherAttributeName}{RGB}{229,0,0}%"#e50000"
\definecolor{entityOtherAttributeNameScss}{RGB}{128,0,0}%"#800000"
\definecolor{invalid}{RGB}{205,49,49}%"#cd3131"
\definecolor{markupBold}{RGB}{0,0,128}%"#000080"
\definecolor{markupHeading}{RGB}{128,0,0}%"#800000"
\definecolor{markupInserted}{RGB}{9,134,88}%"#098658"
\definecolor{markupDeleted}{RGB}{163,21,21}%"#a31515"
\definecolor{markupChanged}{RGB}{4,81,165}%"#0451a5"
\definecolor{punctuationDefinitionListBeginMarkdown}{RGB}{4,81,165}%"#0451a5"
\definecolor{markupInlineRaw}{RGB}{128,0,0}%"#800000"
\definecolor{bracketsOfXmlhtmlTags}{RGB}{128,0,0}%"#800000"
\definecolor{entityNameFunctionPreprocessor}{RGB}{0,0,255}%"#0000ff"
\definecolor{metaPreprocessorString}{RGB}{163,21,21}%"#a31515"
\definecolor{metaPreprocessorNumeric}{RGB}{9,134,88}%"#098658"
\definecolor{metaStructureDictionaryKeyPython}{RGB}{4,81,165}%"#0451a5"
\definecolor{storage}{RGB}{0,0,255}%"#0000ff"
\definecolor{storageType}{RGB}{0,0,255}%"#0000ff"
\definecolor{keywordOperatorNoexcept}{RGB}{0,0,255}%"#0000ff"
\definecolor{metaEmbeddedAssembly}{RGB}{163,21,21}%"#a31515"
\definecolor{stringQuotedDoubleHandlebars}{RGB}{0,0,255}%"#0000ff"
\definecolor{stringRegexp}{RGB}{129,31,63}%"#811f3f"
\definecolor{stringInterpolation}{RGB}{0,0,255}%"#0000ff"
\definecolor{resetJavascriptStringInterpolationExpression}{RGB}{0,0,0}%"#000000"
\definecolor{supportConstantColor}{RGB}{4,81,165}%"#0451a5"
\definecolor{sourceCoffeeEmbedded}{RGB}{229,0,0}%"#e50000"
\definecolor{supportTypePropertyNameJson}{RGB}{4,81,165}%"#0451a5"
\definecolor{keyword}{RGB}{0,0,255}%"#0000ff"
\definecolor{keywordControl}{RGB}{0,0,255}%"#0000ff"
\definecolor{keywordOperator}{RGB}{0,0,0}%"#000000"
\definecolor{keywordOperatorWordlike}{RGB}{0,0,255}%"#0000ff"
\definecolor{keywordOtherUnit}{RGB}{9,134,88}%"#098658"
\definecolor{punctuationSectionEmbeddedEndPhp}{RGB}{128,0,0}%"#800000"
\definecolor{supportFunctionGitRebase}{RGB}{4,81,165}%"#0451a5"
\definecolor{constantShaGitRebase}{RGB}{9,134,88}%"#098658"
\definecolor{coloringOfTheJavaImportAndPackageIdentifiers}{RGB}{0,0,0}%"#000000"
\definecolor{thisSelf}{RGB}{0,0,255}%"#0000ff"
\definecolor{newOperator}{RGB}{175,0,219}%"#AF00DB"
\definecolor{stringLiteral}{RGB}{163,21,21}%"#a31515"
\definecolor{customLiteral}{RGB}{121,94,38}%"#795E26"
\definecolor{numberLiteral}{RGB}{9,134,88}%"#098658"
\definecolor{functionDeclarations}{RGB}{121,94,38}%"#795E26"
\definecolor{typesDeclarationAndReferences}{RGB}{38,127,153}%"#267f99"
\definecolor{typesDeclarationAndReferencesTsGrammarSpecific}{RGB}{38,127,153}%"#267f99"
\definecolor{controlFlowSpecialKeywords}{RGB}{175,0,219}%"#AF00DB"
\definecolor{variableAndParameterName}{RGB}{0,16,128}%"#001080"
\definecolor{constantsAndEnums}{RGB}{0,112,193}%"#0070C1"
\definecolor{objectKeysTsGrammarSpecific}{RGB}{0,16,128}%"#001080"
\definecolor{cssPropertyValue}{RGB}{4,81,165}%"#0451a5"
\definecolor{regularExpressionGroups}{RGB}{209,105,105}%"#d16969"
\definecolor{constantCharacterSetRegexp}{RGB}{129,31,63}%"#811f3f"
\definecolor{keywordOperatorQuantifierRegexp}{RGB}{0,0,0}%"#000000"
\definecolor{keywordControlAnchorRegexp}{RGB}{238,0,0}%"#EE0000"
\definecolor{constantOtherOption}{RGB}{0,0,255}%"#0000ff"
\definecolor{constantCharacterEscape}{RGB}{238,0,0}%"#EE0000"
\definecolor{entityNameLabel}{RGB}{0,0,0}%"#000000"



\newcommand\digitstyle{\color{numberLiteral}}
\makeatletter
\newcommand{\ProcessDigit}[1]
{%
    \ifnum\lst@mode=\lst@Pmode\relax%
    {\digitstyle #1}%
    \else
    #1%
    \fi
}

\lstdefinestyle{mystyle}
{
	% everything between (* *) is a latex command
	escapeinside={\#(*}{*)},
	language = Python,
	basicstyle = {\normalfont\scriptsize\ttfamily\color{variableAndParameterName}},
	commentstyle=\color{comment},
	backgroundcolor = {\color{back-color}},
	numberstyle=\tiny\ttfamily\color{black},
	stringstyle = {\color{stringLiteral}},
	keywordstyle = {\color{controlFlowSpecialKeywords}},
    keywordstyle = [5]{\color{functionDeclarations}},
    keywordstyle = [2]{\color{functionDeclarations}},
	keywordstyle = [6]{\color{black}},
	keywordstyle = [3]{\color{constantLanguage}},
	keywordstyle = [4]{\color{typesDeclarationAndReferences}}, % Imports
	morekeywords = [6]{.},
	morekeywords = [3]{True, False},
	morekeywords = [4]{
        numpy,
        np,
        os,
        pathlib,
        pl,
        enum,
        Enum,
        DataType,
        pyvisa,
        visa,
        time,
        pandas,
        matplotlib,
        pyplot,
        plt,
        pandas,
        pd,
        shutil,
        I\_IN,
        I\_OUT,
        V\_C,
        V\_OUT,
        I\_IN\_AVG,
        I\_OUT\_AVG,
        V\_OUT\_AVG,
        TIME,
        SCREENSHOT\_ALL,
        PLOT,
        SHOW\_PLOT,
        CLEAR\_PREVIOUS,
        NUMPY\_FORMAT,
        },% packages
	morekeywords = [5]{
        exists
        remove
        geomspace,
        astype,
        linspace,
        write,
        unique,
        savetxt,
        Path,
        unlink,
        open\_resource,
        makedirs,
        query\_ascii\_values,
        append,
        len,
        \_append,
        open,
        figure,
        to\_csv,
        show,
        savefig,
        range,
        insert,
        DataFrame,
        sleep,
        close,
        exit,
        query\_binary\_values,
        split,
        plot,
        geomspace,
        sort,
        query,
        path,
        isdir,
        gmtime,
        strftime,
        printz
        },% functions
	literate=
		{0}{{{\ProcessDigit{0}}}}1
        {1}{{{\ProcessDigit{1}}}}1
        {2}{{{\ProcessDigit{2}}}}1
        {3}{{{\ProcessDigit{3}}}}1
        {4}{{{\ProcessDigit{4}}}}1
        {5}{{{\ProcessDigit{5}}}}1
        {6}{{{\ProcessDigit{6}}}}1
        {7}{{{\ProcessDigit{7}}}}1
        {8}{{{\ProcessDigit{8}}}}1
        {9}{{{\ProcessDigit{9}}}}1
        {e0}{{{\ProcessDigit{e0}}}}2
        {e1}{{{\ProcessDigit{e1}}}}2
        {e2}{{{\ProcessDigit{e2}}}}2
        {e3}{{{\ProcessDigit{e3}}}}2
        {e4}{{{\ProcessDigit{e4}}}}2
        {e5}{{{\ProcessDigit{e5}}}}2
        {e6}{{{\ProcessDigit{e6}}}}2
        {e7}{{{\ProcessDigit{e7}}}}2
        {e8}{{{\ProcessDigit{e8}}}}2
        {e9}{{{\ProcessDigit{e9}}}}2
        {.0}{{{\ProcessDigit{.0}}}}2
        {.1}{{{\ProcessDigit{.1}}}}2
        {.2}{{{\ProcessDigit{.2}}}}2
        {.3}{{{\ProcessDigit{.3}}}}2
        {.4}{{{\ProcessDigit{.4}}}}2
        {.5}{{{\ProcessDigit{.5}}}}2
        {.6}{{{\ProcessDigit{.6}}}}2
        {.7}{{{\ProcessDigit{.7}}}}2
        {.8}{{{\ProcessDigit{.8}}}}2
        {.9}{{{\ProcessDigit{.9}}}}2,
	emph={dir,format},          % Custom highlighting
	emphstyle=\color{variableAndParameterName},
    emph = [2]{print},
    emphstyle=[2]{\color{functionDeclarations}}
}

\usepackage{float}
\newfloat{lstfloat}{htbp}{lop}
\floatname{lstfloat}{Listing}
\def\lstfloatautorefname{Listing} % needed for hyperref/auroref

% For easy quotations: \enquote{text}
% This package is very smart when nesting is applied, otherwise textcmds (see below) provides a shorter command
\usepackage[autostyle=true]{csquotes}

% Enable using "`quote"' - see https://tex.stackexchange.com/a/150954/9075
\defineshorthand{"`}{\openautoquote}
\defineshorthand{"'}{\closeautoquote}

% Nicer tables (\toprule, \midrule, \bottomrule)
\usepackage{booktabs}
\usepackage{longtable}
\usepackage{colortbl}
\usepackage{csvsimple}

% Extended enumerate, such as \begin{compactenum}
\usepackage{paralist}

% Bibliopgraphy enhancements
%  - enable \cite[prenote][]{ref}
%  - enable \cite{ref1,ref2}
% Alternative: \usepackage{cite}, which enables \cite{ref1, ref2} only (otherwise: Error message: "White space in argument")

% Doc: http://texdoc.net/natbib
\usepackage[%
  square,        % for square brackets
  comma,         % use commas as separators
  numbers,       % for numerical citations;
  %sort           % orders multiple citations into the sequence in which they appear in the list of references;
  sort&compress  % as sort but in addition multiple numerical citations are compressed if possible (as 3-6, 15);
]{natbib}

% Same fontsize as without natbib
\renewcommand{\bibfont}{\normalfont\footnotesize}

% Enable hyperlinked author names in the case of \citet
% Source: https://tex.stackexchange.com/a/76075/9075
\usepackage{etoolbox}
\makeatletter
\patchcmd{\NAT@test}{\else \NAT@nm}{\else \NAT@hyper@{\NAT@nm}}{}{}
\makeatother

% Farbige Tabellen
% ----------------
% Das Paket colortbl wird inzwischen automatisch durch  geladen
%
% Erweiterte Funktionen innerhalb von Tabellen
% --------------------------------------------
%%% Doc: http://mirror.ctan.org/tex-archive/macros/latex/contrib/multirow/multirow.sty
\usepackage{multirow} % Mehrfachspalten
%
%%% Doc: Documentation inside dtx Package
\usepackage{dcolumn}  % Ausrichtung an Komma oder Punkt

%%% Doc: http://mirror.ctan.org/tex-archive/macros/latex/contrib/supertabular/supertabular.pdf
%\usepackage{supertabular}

%%% Fussnoten/Endnoten ===================================================

% EN: Put footnotes below floats
% DE: Fußnoten unter Gleitumgebungen ("floats") platzieren
% Source: https://tex.stackexchange.com/a/32993/9075
\usepackage{stfloats}
\fnbelowfloat

% EN: Extended support for footnotes
% DE: Fußnoten
%
%\usepackage{dblfnote}  %Zweispaltige Fußnoten
%
% Keine hochgestellten Ziffern in der Fußnote (KOMA-Script-spezifisch):
%\deffootnote[1.5em]{0pt}{1em}{\makebox[1.5em][l]{\bfseries\thefootnotemark}}
%
% Abstand zwischen Fußnoten vergrößern:
%\setlength{\footnotesep}{.85\baselineskip}
%
% EN: Following command disables the separting line of the footnote
% DE: Folgendes Kommando deaktiviert die Trennlinie zur Fußnote
%\renewcommand{\footnoterule}{}
%
%\addtolength{\skip\footins}{\baselineskip} % Abstand Text <-> Fußnote

% DE: Fußnoten immer ganz unten auf einer \raggedbottom-Seite
% DE: fnpos kommt aus dem yafoot package
%\usepackage{fnpos}
%\makeFNbelow
%\makeFNbottom

% TODO (and comment) configuration
%
% - \todo (from todo, easy-todo, todonotes) / \TODO (from fixmetodonotes) - for "normal" TODOs
% - \todofix - "important" TODOs
%
% - \textcomment - highlights text and has a hover comment
% - \sidecomment - just puts a comment to the side. Note: \comment MUST NOT be used as command name, it is already defined by much packages (mathdesign, mindflow, verbatim, and others)
%
% - \missingfigure
%
% - \textmarker
% - \modified
% - \change      - adresses a review comment

% Enable nice comments
\usepackage{pdfcomment}

\newcommand{\textcomment}[2]{\colorbox{yellow!60}{#1}\pdfcomment[color={0.234 0.867 0.211},hoffset=-6pt,voffset=10pt,opacity=0.5]{#2}}

% Small PDF comment
% 1. Parameter: Comment
\newcommand{\sidecomment}[1]{\pdfcomment[color={0.045 0.278 0.643},voffset=4pt,icon=Note]{#1}}
% Disabled variant - for the final PDF
%\newcommand{\sidecomment}[1]{}

\newcommand{\todo}[1]{TODO!\sidecomment{#1}}

% Änderungen
%
% 1. Parameter: Review-Kommentar
% 2. Parameter: Neuer Text
\newcommand{\change}[2]{{\color{red}#2}\pdfcomment[color={0.234 0.867 0.211},voffset=8pt,opacity=0.5]{#1}}
% Disabled variant - for the final PDF
%\newcommand{\change}[2]{#2}

% Define default commands
\makeatletter
\@ifundefined{missingfigure}{\newcommand{\missingfigure}{... missing figure ...}}{}
\@ifundefined{textcomment}{\newcommand{\textcomment}[2]{#1 \todo{#2}}}{}
\@ifundefined{sidecomment}{\newcommand{\sidecomment}[1]{\marginpar{#1}}}{}
\@ifundefined{todo}{\newcommand{\todo}[1]{\sidecomment{#1}}}{}
\@ifundefined{TODO}{\newcommand{\TODO}[1]{\todo{#1}}}{}
\@ifundefined{todofix}{\newcommand{\todofix}[1]{\todo{#1}}}{}
\@ifundefined{change}{\newcommand{\change}[2]{#1 $\rightarrow$ #2}}{}
\makeatother

% Textmarker (Textfarbe rot)
\newcommand{\textmarker}[1]{{\color{red} #1}\xspace}

% Modified (Text blau)
\newcommand{\modified}[1]{{\color{blue!60!black} #1}\xspace}

\usepackage[group-minimum-digits=4,per-mode=fraction]{siunitx}

% Enable that parameters of \cref{}, \ref{}, \cite{}, ... are linked so that a reader can click on the number an jump to the target in the document

\usepackage{hyperref}
\usepackage{bookmark}
\usepackage{ifdraft}

% Enable hyperref without colors and without bookmarks
\ifoptionfinal{
    \hypersetup{
        final,
        colorlinks=true,       % Links erhalten Farben statt Kaeten
        raiselinks=true,       % calculate real height of the link
         allcolors=black,
        pdfstartview=Fit,
        breaklinks=true,       % Links ueberstehen Zeilenumbruch
        hypertexnames=false,   % Fix jumping to algorithm line - http://tex.stackexchange.com/a/156404/9075
    }
}{
    \hypersetup{
        hidelinks,
        colorlinks=true,       % Links erhalten Farben statt Kaeten
        raiselinks=true,       % calculate real height of the link
        pdfstartview=Fit,
        breaklinks=true,       % Links ueberstehen Zeilenumbruch
        hypertexnames=false,   % Fix jumping to algorithm line - http://tex.stackexchange.com/a/156404/9075
    }
}


% Enable correct jumping to figures when referencing
\usepackage[all]{hypcap}

%\usepackage[caption=false,font=footnotesize]{subfig}

% Alternative for making subfigures:
% Part of the caption package. See http://www.ctan.org/pkg/caption
% (subfigure is outdated. subfig is maintained, but subcaption is better)
% See: http://tex.stackexchange.com/questions/13625/subcaption-vs-subfig-best-package-for-referencing-a-subfigure
\usepackage[hypcap=true]{subcaption}

\usepackage[incolumn]{mindflow}

% Extensions for references inside the document (\cref{fig:sample}, ...)
% Enable usage \cref{...} and \Cref{...} instead of \ref: Type of reference included in the link
% That means, "Figure 5" is a full link instead of just "5".
\usepackage[capitalise,nameinlink,noabbrev]{cleveref}

\crefname{listing}{Listing}{Listings}
\Crefname{listing}{Listing}{Listings}
\crefname{lstlisting}{Listing}{Listings}
\Crefname{lstlisting}{Listing}{Listings}

\usepackage{lipsum}
%
%% For demonstration purposes only
%% These packages can be removed when all examples have been deleted
%\usepackage[math]{blindtext}
%\usepackage{mwe}
%\usepackage[realmainfile]{currfile}
%\usepackage{tcolorbox}
%\tcbuselibrary{listings}

% Allows for defining commands that don't eat spaces.
\usepackage{xspace}
% Adds compatibility to \xspace und \enquote
\makeatletter
\xspaceaddexceptions{\grqq \grq \csq@qclose@i \} }
\makeatother

\newcommand{\eg}{e.g.,\ }
\newcommand{\ie}{i.e.,\ }

% Enable hyphenation at other places as the dash.
% Example: applicaiton\hydash specific
\makeatletter
\newcommand{\hydash}{\penalty\@M-\hskip\z@skip}
% Definition of "= taken from http://mirror.ctan.org/macros/latex/contrib/babel-contrib/german/ngermanb.dtx
\makeatother

% Add manual adapted hyphenation of English words
% See https://ctan.org/pkg/hyphenex and https://tex.stackexchange.com/a/22892/9075 for details
\input{ushyphex}

% correct bad hyphenation here
\hyphenation{
  op-tical net-works semi-conduc-tor
  % May not be hypphenated
  AROMA TOSCA BPMN OASIS OMG DMTF IT DevOps
}

% 🇩🇪 wird fuer Tabellen benötigt (z.B. >{centering\RBS}p{2.5cm} erzeugt einen zentrierten 2,5cm breiten Absatz in einer Tabelle
\newcommand{\RBS}{\let\\=\tabularnewline}

% 🇺🇸 To avoid issues with Springer's \mathplus. See also http://tex.stackexchange.com/q/212644/9075
\providecommand\mathplus{+}

% 🇺🇸 from hmks makros.tex - \indexify
\newcommand{\toindex}[1]{\index{#1}#1}

% 🇩🇪 Tipp aus "The Comprehensive LaTeX Symbol List"
\newcommand{\dotcup}{\ensuremath{\,\mathaccent\cdot\cup\,}}

% 🇩🇪 Anstatt $|x|$ $\abs{x}$ verwenden. Die Betragsstriche skalieren automatisch, falls "x" etwas größer sein sollte...
\newcommand{\abs}[1]{\left\lvert#1\right\rvert}

% 🇩🇪 Seitengrößen - Gegen Schusterjungen und Hurenkinder...
\newcommand{\largepage}{\enlargethispage{\baselineskip}}
\newcommand{\shortpage}{\enlargethispage{-\baselineskip}}

\newcommand{\initialism}[1]{%
  \textlcc{#1}\xspace%
}
\newcommand{\OMG}{\initialism{OMG}}
\newcommand{\BPEL}{\initialism{BPEL}}
\newcommand{\BPMN}{\initialism{BPMN}}
\newcommand{\UML}{\initialism{UML}}

%introduce \powerset - hint by http://matheplanet.com/matheplanet/nuke/html/viewtopic.php?topic=136492&post_id=997377
\DeclareFontFamily{U}{MnSymbolC}{}
\DeclareSymbolFont{MnSyC}{U}{MnSymbolC}{m}{n}
\DeclareFontShape{U}{MnSymbolC}{m}{n}{
	<-6>    MnSymbolC5
	<6-7>   MnSymbolC6
	<7-8>   MnSymbolC7
	<8-9>   MnSymbolC8
	<9-10>  MnSymbolC9
	<10-12> MnSymbolC10
	<12->   MnSymbolC12%
}{}
\DeclareMathSymbol{\powerset}{\mathord}{MnSyC}{180}


\ifpdftex
  % Enable copy and paste of text from the PDF
  % Only required for pdflatex. It "just works" in the case of lualatex.
  % Alternative: cmap or mmap package
  % mmap enables mathematical symbols, but does not work with the newtx font set
  % See: https://tex.stackexchange.com/a/64457/9075
  % Other solutions outlined at http://goemonx.blogspot.de/2012/01/pdflatex-ligaturen-und-copynpaste.html and http://tex.stackexchange.com/questions/4397/make-ligatures-in-linux-libertine-copyable-and-searchable
  % Trouble shooting outlined at https://tex.stackexchange.com/a/100618/9075
  %
  % According to https://tex.stackexchange.com/q/451235/9075 this is the way to go
  \input{glyphtounicode}
  \pdfgentounicode=1
\fi

\usepackage{siunitx}




\usepackage{circuitikz}
\usepackage{nicematrix}
\usepackage{svg}
\usepackage{enumitem}



\usepackage{float}
\usepackage{aliascnt}
\newaliascnt{eqfloat}{equation}
\newfloat{eqfloat}{h}{eqflts}
\floatname{eqfloat}{Equation}

\newcommand*{\ORGeqfloat}{}
\let\ORGeqfloat\eqfloat
\def\eqfloat{%
	\let\ORIGINALcaption\caption
	\def\caption{%
		\addtocounter{equation}{-1}%
		\ORIGINALcaption
	}%
	\ORGeqfloat
}

\renewcommand{\thetable}{\arabic{table}}


\DeclareCaptionFormat{custom}
{%
    \footnotesize{#1#2#3}%\textbf{#1#2}\textit{\small #3}
}
\captionsetup{format=custom}

\usepackage{standalone}
\usepackage{multicol}